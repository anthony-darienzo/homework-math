\documentclass[countbysections]{./homework-math}

\title{1.4 Injective Modules}
\author{Roger Godement}
\date{\normalfont{from~\textit{Topologie alg\'ebrique et Th\'eorie de
faisceaux}}}
\contact{ca. 1964}

\newcommand{\T}{\mathbb{T}}

\begin{document}
	
\maketitle%

\smalltableofcontents%

\section{Test}

An \(A\)-module \(L\) is \emph{injective} if for every exact sequence%
\[%
	0 \to X' \to X \to X'' \to 0,
\]%
the corresponding sequence%
\[%
	0 \leftarrow \Hom\pwrap{X',L} \leftarrow \Hom\pwrap{X,L} \leftarrow
	\Hom\pwrap{X'',L} \leftarrow 0
\]%
is exact as well. Alternatively \(L\) is exact in the case that the following
condition holds:

\noindent(INJ) : \emph{%
	Let \(j: X' \to X\) be an injective morphism. For any morphism \(f': X' \to
	L\), there exists a morphism \(f: X \to L\) such that \(f' = f \circ j\).
}%

This means that any morphism into \(L\) from a submodule of \(X\) extends to a
morphism from \(X\) into \(L\).

\begin{theorem}[1.4.1]
	For a left \(A\)-module to be injective, it is necessary and sufficient that,
	for any left ideal \(I \triangleleft A\) and any morphism \(f: I \to L\),
	there exists an \(x \in L\) such that \(f\pwrap\lambda = \lambda \cdot x\)
	for any \(\lambda \in I\).
\end{theorem}

This condition is necessary since \(f\) should extend to a morphism \(A \to
L\).

Conversely, suppose the above condition holds, and consider \(X\) a module,
\(X'\) a submodule of \(X\), and \(f\) a morphism \(X' \to L\). Consider the
pairs \(\pwrap{Y,g}\) where \(Y\) is a submodule of \(X\) containing \(X'\) and
\(g\) a morphism \(Y \to L\) which extends \(f\). We see by Zorn's Lemma that
\(f\) admits a maximal extension \(\pwrap{Y,g}\). If \(Y\) were different from
\(X\), we could find an \(x \in X\) outside of \(Y\), a left ideal \(I
\triangleleft A\) (the set of \(\lambda \in A\) such that \(\lambda x \in Y\)),
and a morphism \(h: I \to L\), given by \(\lambda \mapsto g(\lambda\cdot x)\).
By the hypothesis, there exists an element \(u\) of \(L\) such that the
relation \(\lambda x \in Y\) implies \(g\pwrap{\lambda x} = \lambda u\); but
then \(g\) could extend to a submodule of \(X\) generated by \(Y\) and \(x\): a
contradiction.

For example, if the base ring \(A\) is a principal ideal domain, then \(L\) is
injective if and only if, for every \(\lambda \neq 0\), the endomorphism \(x
\mapsto \lambda x\) of \(L\) is surjective.

\begin{theorem*}[1.2.2]
	Every \(A\)-module is a submodule of an injective \(A\)-module.
\end{theorem*}

To see this, we consider the ring \(Z\) of integers, the additive group \(\Q\)
of rational numbers, and the \(\Z\)-module%
\[%
	\T \eqdef \Q/\Z.
\]%
This module is injective by the preceding theorem, and it is clear that any
cyclic abelian group (of finite or infinite order) maps nontrivially\footnote{%
	Godement uses the word ``se plonge'' which suggests that any cyclic abelian
	group \emph{embeds} into \(\T\), which is clearly false (how would the
	integers embed?). However, Godement's proof only needs the existence of a
	nontrivial morphism.
} into \(\T\).

Let \(L\) be a left \(A\)-module, and consider%
\[%
	\hw{L} \eqdef \Hom_{\Z}\pwrap{L,\T};
\]%
this is in an evident manner a right \(A\)-module; by the same logic,%
\[%
	\hw{\hw{L}} \eqdef \Hom_{\Z}\pwrap{\hw{L},\T}
\]%
is a left \(A\)-module, and there is a canonical morphism%
\[%
	L \to \hw{\hw{L}}
\]%
defined in an obvious way.\footnote{%
	This is the \emph{evaluation} map \(x \mapsto \pwrap{\phi \mapsto \phi(x)}\).
} We now show that this morphism is injective. To that end consider a
nontrivial \(x \in L\), we can construct a morphism \(f: L \to \T\) such that
\(f(x) \neq 0\): let \(G\) be the cyclic group generated by \(x\) inside \(L\);
\(G\) maps nontrivially into \(\T\), let \(f\) be a nontrivial map defined on
\(G\)---since \(\T\) is an injective \(\Z\)-module, \(f\) extends to \(L\), as
desired.

Now we show that \emph{if \(L\) is projective, then \(\hw{L}\) is injective};
to that end let \(X\) be a right \(A\)-module and \(f\) a morphism from a
submodule \(X'\) of \(X\) into the \(A\)-module \(\hw{L}\). By duality we have
a morphism \(\hw{f}: \hw{\hw{L}} \to \hw{X'}\) and in particular, since \(L\)
embeds into \(\hw{\hw{L}}\), we have a morphism \(L \to \hw{X}\), which in turn
implies by duality a morphism \(\hw{\hw{X}} \to \hw{L}\) which extends \(f\),
hence \emph{a fortiori} extends the original \(f\) to \(X\).

To complete the proof, we represent \(\hw{L}\) as a quotient of a projective
module \(F\); hence \(\hw{\hw{L}}\), and \emph{a fortiori} \(L\), embeds into
\(\hw{F}\), which is injective by what we just observed---which completes the
proof.

We leave the reader the job of proving, with the help of the preceding result,
that \emph{for an \(L\) module to be injective, it is necessary and sufficient
that it is a direct factor in every module which contains \(L\)}.

\end{document}
